\documentclass[12pt]{article}
%NOTE: This report format is 

\newcommand{\reporttitle}{Rapport de Projet : Régulation d’un bâtiment thermiquement actif}
\newcommand{\reportauthorOne}{Benjamin BOCK}
\newcommand{\cidOne}{S2304467}
\newcommand{\reportauthorTwo}{Odayfa DAKIR}
\newcommand{\cidTwo}{S23XXXXX}
\newcommand{\reportauthorThree}{Yazan SALOUM}
\newcommand{\cidThree}{S23XXXXX}
\newcommand{\reporttype}{Coursework}
\bibliographystyle{plain}

% include files that load packages and define macros
\usepackage{fontspec}
\usepackage{newtxtext,newtxmath} % Times New Roman pour texte et maths



% Packages utiles (ajoutez ceux dont vous avez besoin)
\usepackage[letterpaper,hmargin=2.8cm,vmargin=2.0cm,includeheadfoot]{geometry}
\usepackage{textpos}
\usepackage{natbib}
\usepackage{stackengine}
\usepackage{tabularx,longtable,multirow,subfigure,caption}%hangcaption
\usepackage{fncylab} %formatting of labels
\usepackage{fancyhdr}
\usepackage{color}
\usepackage[tight,ugly]{units}
\usepackage{url}
\usepackage{float}
\usepackage[english]{babel}
\usepackage{amsmath}
\usepackage{graphicx}
\usepackage[colorinlistoftodos]{todonotes}
\usepackage{dsfont}
\usepackage{epstopdf} % automatically replace .eps with .pdf in graphics
\usepackage{backref}
\usepackage{array}
\usepackage{etoolbox}
\usepackage{enumerate} % for numbering with [a)] format 
\usepackage{tcolorbox}
\usepackage{graphicx} % Pour insérer des images
\usepackage{tocloft}  % Pour personnaliser la liste des figures
\usepackage[english,french]{babel}
\usepackage[T1]{fontenc}
\usepackage{listings} % Pour améliorer l'affichage du code
\usepackage{enumitem}
\usepackage{listings} % Pour afficher du code
\usepackage{xcolor}   % Pour colorer le code
\usepackage{longtable}
\usepackage{gensymb}
\usepackage{mathtools}

\newcommand{\makefootnotelist}[1]{%
    \parbox{0.8\textwidth} {%
        \footnotesize{%
            \renewcommand*{\do}[1]{##1\\}%
            \dolistcsloop{#1}}}}%
\newcommand{\fancyfootnote}[1]{%
    \footnotemark{}%
    \def\listname{footlist\thepage}%
    \def\n{$^{\the\numexpr\value{footnote}}$}
    \ifcsdef{\listname}%
        {\listcseadd{\listname}{\n\ #1}}%
        {\csedef{\listname}{}%
        \listcseadd{\listname}{\n\ #1}}%
    \fancypagestyle{fancyfootnote}{%
        \fancyfoot[LO,RE]{\makefootnotelist{\listname}}%
        \fancyfoot[RO,LE]{\thepage}%
        \fancyfoot[C]{}%
    }\thispagestyle{fancyfootnote}}%

\fancypagestyle{plain}{%
  \fancyfoot[C]{\thepage}
  \renewcommand{\headrulewidth}{0.4pt}
  \renewcommand{\footrulewidth}{0.4pt}
}


\captionsetup[table]{name=Tableau}  % Remplacer "Table" par "Tableau"

% Définition du style Python
\lstdefinelanguage{Python}{
  keywords={from, import, as, def, return, if, elif, else, for, while, break, continue, in, not, and, or, pass, lambda, with, yield, try, except, finally, class, True, False, None},
  keywordstyle=\color{blue}\bfseries,
  comment=[l]{\#},
  commentstyle=\color{gray},
  string=[b]",
  stringstyle=\color{red},
  morestring=[b]',
  basicstyle=\ttfamily\small,
  showstringspaces=false
}


% table of content
\renewcommand{\tableofcontentsname}{Table of Contents}

% table of figures
\renewcommand{\listfigurename}{Table of Figures}

% various theorems
\usepackage{ntheorem}
\theoremstyle{break}
\newtheorem{lemma}{Lemma}
\newtheorem{theorem}{Theorem}
\newtheorem{remark}{Remark}
\newtheorem{definition}{Definition}
\newtheorem{proof}{Proof}

% example-environment
\newenvironment{example}[1][]
{ 
\vspace{4mm}
\noindent\makebox[\linewidth]{\rule{\hsize}{1.5pt}}
\textbf{Example #1}\\
}
{ 
\noindent\newline\makebox[\linewidth]{\rule{\hsize}{1.0pt}}
}

\setlength{\parindent}{0em}  % indentation of paragraph

\setlength{\headheight}{14.5pt}
\pagestyle{fancy}
\fancyfoot[ER,OR]{\thepage}%Page no. in the left on
                                %odd pages and on right on even pages
\fancyfoot[OC,EC]{\sffamily }
\renewcommand{\headrulewidth}{0.1pt}
\renewcommand{\footrulewidth}{0.1pt}
\captionsetup{margin=10pt,font=small,labelfont=bf}

%--- chapter heading
\def\@makechapterhead#1{%
  \vspace*{10\p@}%
  {\parindent \z@ \raggedright
    \interlinepenalty\@M
    \Huge \bfseries 
    \thechapter \space\space #1\par\nobreak
    \vskip 30\p@
  }}

%---chapter heading for \chapter*  
\def\@makeschapterhead#1{%
  \vspace*{10\p@}%
  {\parindent \z@ \raggedright
    \interlinepenalty\@M
    \Huge \bfseries  
    #1\par\nobreak
    \vskip 30\p@
  }}

% %%%%%%%%%%%%% boxit
\def\Beginboxit
   {\par
    \vbox\bgroup
	   \hrule
	   \hbox\bgroup
		  \vrule \kern1.2pt %
		  \vbox\bgroup\kern1.2pt
   }

\def\Endboxit{%
			      \kern1.2pt
		       \egroup
		  \kern1.2pt\vrule
		\egroup
	   \hrule
	 \egroup
   }	

\newenvironment{boxit}{\Beginboxit}{\Endboxit}
\newenvironment{boxit*}{\Beginboxit\hbox to\hsize{}}{\Endboxit}

\allowdisplaybreaks

\makeatletter
\newcounter{elimination@steps}
\newcolumntype{R}[1]{>{\raggedleft\arraybackslash$}p{#1}<{$}}
\def\elimination@num@rights{}
\def\elimination@num@variables{}
\def\elimination@col@width{}
\newenvironment{elimination}[4][0]
{
    \setcounter{elimination@steps}{0}
    \def\elimination@num@rights{#1}
    \def\elimination@num@variables{#2}
    \def\elimination@col@width{#3}
    \renewcommand{\arraystretch}{#4}
    \start@align\@ne\st@rredtrue\m@ne
}
{
    \endalign
    \ignorespacesafterend
}
\newcommand{\eliminationstep}[2]
{
    \ifnum\value{elimination@steps}>0\leadsto\quad\fi
    \left[
        \ifnum\elimination@num@rights>0
            \begin{array}
            {@{}*{\elimination@num@variables}{R{\elimination@col@width}}
            |@{}*{\elimination@num@rights}{R{\elimination@col@width}}}
        \else
            \begin{array}
            {@{}*{\elimination@num@variables}{R{\elimination@col@width}}}
        \fi
            #1
        \end{array}
    \right]
    & 
    \begin{array}{l}
        #2
    \end{array}
    &%                                    moved second & here
    \addtocounter{elimination@steps}{1}
}
\makeatother

%% Fast macro for column vectors
\makeatletter  
\def\colvec#1{\expandafter\colvec@i#1,,,,,,,,,\@nil}
\def\colvec@i#1,#2,#3,#4,#5,#6,#7,#8,#9\@nil{% 
  \ifx$#2$ \begin{bmatrix}#1\end{bmatrix} \else
    \ifx$#3$ \begin{bmatrix}#1\\#2\end{bmatrix} \else
      \ifx$#4$ \begin{bmatrix}#1\\#2\\#3\end{bmatrix}\else
        \ifx$#5$ \begin{bmatrix}#1\\#2\\#3\\#4\end{bmatrix}\else
          \ifx$#6$ \begin{bmatrix}#1\\#2\\#3\\#4\\#5\end{bmatrix}\else
            \ifx$#7$ \begin{bmatrix}#1\\#2\\#3\\#4\\#5\\#6\end{bmatrix}\else
              \ifx$#8$ \begin{bmatrix}#1\\#2\\#3\\#4\\#5\\#6\\#7\end{bmatrix}\else
                 \PackageError{Column Vector}{The vector you tried to write is too big, use bmatrix instead}{Try using the bmatrix environment}
              \fi
            \fi
          \fi
        \fi
      \fi
    \fi
  \fi 
}  
\makeatother

\robustify{\colvec} % various packages needed for maths etc.
\input{notation} % short-hand notation and macros


%%%%%%%%%%%%%%%%%%%%%%%%%%%%

\begin{document}
\sloppy % Permet plus de tolérance d'espace entre les mots

% front page
% Last modification: 2016-09-29 (Marc Deisenroth)
% Modification for UW: 2017-05-22 (jphickey)
\begin{titlepage}

\newcommand{\HRule}{\rule{\linewidth}{0.5mm}} % Defines a new command for the horizontal lines, change thickness here


%----------------------------------------------------------------------------------------
%	LOGO SECTION
%----------------------------------------------------------------------------------------



\begin{center} % Center remainder of the page

%----------------------------------------------------------------------------------------
%	HEADING SECTIONS
%----------------------------------------------------------------------------------------

\includegraphics[width = 15cm]{./figures/uliege_faculte_sciencesappliquees_logo_rvb}\\[1.5cm] 
\textbf{\textsc{\Large PROJ0001-1 Introduction \\
aux méthodes numériques et projet}}\\[1.0cm] 
\textsc{\Large Université de Liège}\\[0.5cm] 
\textsc{\large Faculté des Sciences Appliquées}\\[0.45cm] 
\textsc{\large Année académique 2024-2025}\\[0.45cm] 


%----------------------------------------------------------------------------------------
%	TITLE SECTION
%----------------------------------------------------------------------------------------

\HRule \\[0.4cm]
{ \huge \bfseries \reporttitle}\\ % Title of your document
\HRule \\[1.5cm]
\end{center}
%----------------------------------------------------------------------------------------
%	AUTHOR SECTION
%----------------------------------------------------------------------------------------

%\begin{minipage}{0.4\hsize}

\begin{center}
    \begin{minipage}{0.45\linewidth} % Bloc à gauche
        \raggedright % Aligné à gauche
        \normalsize \textit{Professeurs:} \\
            \small  Olivier BRULS \\
                    Quentin LOUVEAUX \\
                    Frédéric NGUYEN 
    \end{minipage}
    \begin{minipage}{0.45\linewidth} % Bloc à droite
        \raggedleft % Aligné à droite
        \normalsize \textit{Auteurs:} \\
        \begin{small}
            \reportauthorOne~(\cidOne)\\
            \reportauthorTwo~(\cidTwo)\\
            \reportauthorThree~(\cidThree)\\
        \end{small}
    \end{minipage}
\end{center}


\vspace{4cm}
\makeatletter
Liège, le \today

\vfill % Fill the rest of the page with whitespace



\makeatother

\end{titlepage}




%%%%%%%%%%%%%%%%%%%%%%%%%%% table of content
%If a table of content is needed, simply uncomment the following lines
\renewcommand{\contentsname}{Table des matières}
\renewcommand{\listfigurename}{Table des figures}
\renewcommand{\listtablename}{Liste des tableaux}

\tableofcontents
\newpage
\listoffigures
\listoftables
\newpage


%%%%%%%%%%%%%%%%%%%%%%%%%%%% Main document
\renewcommand\thesection{0} % Définit le numéro de section à 0
\section{Introduction}

Le projet s'inscrit dans le cadre de l'optimisation énergétique des bâtiments et du domaine de la construction en générale. Le but étant d'apporter aux étudiants en première année de bachelier en ingénierie civile des outils modernes de modélisation numérique. Ce projet en veut pour preuve l'utilisation du langage de programmation Python, de loin le langage le plus populaire en 2025. Pour plus d'informations sur ce sujet, voir la source \cite{Tiobe2025}. \\
Le sujet de ce projet tourne autour du contrôle de transmission de chaleur aux travers des différents parois qui composent les murs d'une bâtiment thermiquement actif. \\
La suite de ce rapport sera consacré à la réponse aux questions posées dans le fichier \texttt{Enonce\_2025.pdf}, où la numération des sections et leurs sous-sections correspondent, respectivement, aux numéros des questions répondues.

\renewcommand\thesection{\arabic{section}} % Rétablit la numérotation normale
\setcounter{section}{0} % Remet le compteur à 0 pour que la prochaine section soit 1

\section{Recherche de racines}

Cette première question vise à modéliser deux fonctions : \texttt{secante(f,x0,x1,tol)} et \texttt{bissection(f,x0,x1,tol)}. Ces fonctions jouent le même rôle, qui est de rechercher la racine d'une fonction $f(x)$ donnée, mais ont des fonctionnements différents. Ainsi, selon le cas, c'est à l'étudiant à choisir, judicieusement, laquelle des deux se voit être la plus encline à être utilisée dans le contexte rencontré. Ces fonctions doivent retourner un objet de type \textit{list} avec deux valeurs, \texttt{x} et \texttt{statut}.

\subsection{Bissection}
    La fonction \texttt{bissection} se base sur la méthode d'analyse numérique éponyme de recherche de racine. Celle-ci prend en arguments la fonction $f(x)$ elle-même, deux points d'abscisses différents et la tolérance d'erreur maximale admissible par l'algorithme. Afin d'assurer le bon fonctionnement de l'algorithme, il vient de vérifier les points suivants :
\begin{enumerate}[label=\roman*.]
    \item Hypothèses de la méthode : 
        \begin{itemize}
            \item 
                Les images des deux points initiaux, \texttt{x0} et \texttt{x1}, doivent être de signes contraires. Ceci peut-être vérifié en testant le signe de l'expression $f(x_1)f(x_0)$. Ainsi, viennent deux cas possibles :
                
                \begin{itemize}
                    \item Cas 1, $f(x_1)f(x_0) < 0$ : \vspace{2mm} \\ 
                    Les deux images sont de signes contraires, l'hypothèse est donc vérifiée. \vspace{2mm} 
                    \item Cas 2, $f(x_1)f(x_0) > 0$ : \vspace{2mm} \\
                    Les deux images sont de mêmes signes, la méthode de la bissection n'est pas possible sur l'intervalle considéré. On retourne \texttt{statut = 1}.
                \end{itemize}
            \item
                La fonction $f(x)$ doit être continue dans l'intervalle $[x_0 , x_1]$, soit $f \in C_0([x_0,x1])$. Malheureusement, aucune fonction de la bibliothèque standard Python ne le permet, tout comme les bibliothèques externes \texttt{numpy}, \texttt{scipy} et \texttt{matplotlib}. Il est ainsi impossible de prévoir le caractère de la fonction sur l'intervalle considéré. Supposons pour le projet que la fonction $f(x)$ dont on cherche la racine est bien continue sur l'intervalle. Nous verrons par la suite que cette hypothèse est acceptable car nous étudions un phénomène physique d'évolution de température dans un espace, en trois dimensions, continu, de sorte que la fonction est bien continue sur l'intervalle et l'hypothèse est vérifiée.

        \end{itemize}
    \item 
    Nombre d'appels de la fonction $f(x)$ :
        \begin{itemize}
            \item Une attention particulière a été portée à ce sujet. En effet, l'algorithme n'appelle qu'une seule fois $f(x)$ par itération. Ce qui est logique car la méthode de la bissection consiste à calculer un nouveau point pour chaque nouvelle itération, en se rapprochant, pas à pas, de la solution exacte.
        \end{itemize}
\end{enumerate}

\subsection{Sécante}
    La fonction \texttt{secante} se base sur la méthode d'analyse numérique éponyme de recherche de racine. Son but est le même que la fonction précédente mais possède son lot de différence auxquelles le programmeur doit tenir rigueur. Tout comme \texttt{bissection}, celle-ci prend en arguments la fonction $f(x)$ elle-même, deux points d'abscisses différents et la tolérance d'erreur maximale admissible par l'algorithme. Afin d'assurer le bon fonctionnement de l'algorithme, il vient de vérifier les points suivants :
\begin{enumerate}[label=\roman*.]
    \item Hypothèses de la méthode : 
        \begin{itemize}
            \item Celles-ci sont exactement les mêmes que pour la bissection. On procède de manière analogue que précédemment.
        \end{itemize}
        
    \item 
    Nombre d'appels de la fonction $f(x)$ : 
        \begin{itemize}
            \item Idem bissection
        \end{itemize}
        
\end{enumerate}

\subsection{Comparaison des deux fonctions}

Le tableau \ref{tab:comparaison} met en évidence les différences entre les deux fonctions \texttt{bissection} et \texttt{secante}. \\
Notons que pour les deux fonctions, il est possible de déterminer une approximation du nombre d'itérations maximales en fonction de la tolérance \texttt{tol} et des points initiaux \texttt{x0}, \texttt{x1}. Ceci peut-être donné avec la formule

\begin{equation}
    k_{\text{max}} = \log_2 \left( \frac{|x_1 - x_0|}{2 \cdot \text{tol}} \right).
    \label{eq:k_max}
\end{equation}


\begin{table}[b!]
    
    \begin{tabular}{| m{6em} | m{6cm}| m{6.2cm} |}
    \hline
    \textbf{Critère} & \textbf{Bissection} & \textbf{Sécante} \\
    \hline
    \textbf{Vitesse} & Lente, surtout pour des intervalles larges : 48 µs. & Plus rapide que la bissection : 32 µs \\
    \hline
    \textbf{Ordre de convergence} & Linéaire : 1 & Superlinéaire : \varphi \approx 1.618 \\
    \hline
    \textbf{Robustesse} & Très robuste : la méthode converge toujours si les conditions sont remplies. & Moins robuste : peut échouer si les points initiaux sont mal choisis ou si la fonction n'est pas bien définie autour des points initiaux. \\
    \hline
    \textbf{Exigence des conditions initiales} & Nécessite seulement deux bornes avec un signe opposé. & Nécessite deux points initiaux distincts et un bon choix pour garantir la convergence. \\
    \hline
    \end{tabular}
    \caption{Comparaison des fonctions \texttt{bissection} et \texttt{secante}}
    \label{tab:comparaison}

\end{table}

\section{Pertes et gains de chaleur}


\newpage
\bibliography{mybib}


\end{document}
%%% Local Variables: 
%%% mode: latex
%%% TeX-master: t
%%% End: 
